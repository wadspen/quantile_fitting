%% 
%% Copyright 2007-2024 Elsevier Ltd
%% 
%% This file is part of the 'Elsarticle Bundle'.
%% ---------------------------------------------
%% 
%% It may be distributed under the conditions of the LaTeX Project Public
%% License, either version 1.3 of this license or (at your option) any
%% later version.  The latest version of this license is in
%%    http://www.latex-project.org/lppl.txt
%% and version 1.3 or later is part of all distributions of LaTeX
%% version 1999/12/01 or later.
%% 
%% The list of all files belonging to the 'Elsarticle Bundle' is
%% given in the file `manifest.txt'.
%% 
%% Template article for Elsevier's document class `elsarticle'
%% with harvard style bibliographic references

\documentclass[preprint,12pt,authoryear]{elsarticle}

%% Use the option review to obtain double line spacing
%% \documentclass[authoryear,preprint,review,12pt]{elsarticle}

%% Use the options 1p,twocolumn; 3p; 3p,twocolumn; 5p; or 5p,twocolumn
%% for a journal layout:
%% \documentclass[final,1p,times,authoryear]{elsarticle}
%% \documentclass[final,1p,times,twocolumn,authoryear]{elsarticle}
%% \documentclass[final,3p,times,authoryear]{elsarticle}
%% \documentclass[final,3p,times,twocolumn,authoryear]{elsarticle}
%% \documentclass[final,5p,times,authoryear]{elsarticle}
%% \documentclass[final,5p,times,twocolumn,authoryear]{elsarticle}

%% For including figures, graphicx.sty has been loaded in
%% elsarticle.cls. If you prefer to use the old commands
%% please give \usepackage{epsfig}

%% The amssymb package provides various useful mathematical symbols
% \usepackage{amssymb}
%% The amsmath package provides various useful equation environments.
% \usepackage{amsmath}
\usepackage{amsmath,amsthm,amssymb,scrextend,mathtools}
\usepackage{hyperref}
\usepackage{newtxmath} %for making indicator function
\usepackage{booktabs}
% \RequirePackage[colorlinks,citecolor=blue,urlcolor=blue,backref=page,backref=page]{hyperref}
%% The amsthm package provides extended theorem environments
%% \usepackage{amsthm}

%% The lineno packages adds line numbers. Start line numbering with
%% \begin{linenumbers}, end it with \end{linenumbers}. Or switch it on
%% for the whole article with \linenumbers.
%% \usepackage{lineno}

\newtheorem{theorem}{Theorem}
\newtheorem{lemma}{Lemma}
\newtheorem{definition}{Definition}
\usepackage{natbib}
\usepackage{adjustbox}
\usepackage{multirow}

\journal{Computational Statistics and Data Analysis}

\begin{document}

% \begin{frontmatter}

%% Title, authors and addresses

%% use the tnoteref command within \title for footnotes;
%% use the tnotetext command for theassociated footnote;
%% use the fnref command within \author or \affiliation for footnotes;
%% use the fntext command for theassociated footnote;
%% use the corref command within \author for corresponding author footnotes;
%% use the cortext command for theassociated footnote;
%% use the ead command for the email address,
%% and the form \ead[url] for the home page:
%% \title{Title\tnoteref{label1}}
%% \tnotetext[label1]{}
%% \author{Name\corref{cor1}\fnref{label2}}
%% \ead{email address}
%% \ead[url]{home page}
%% \fntext[label2]{}
%% \cortext[cor1]{}
%% \affiliation{organization={},
%%            addressline={}, 
%%            city={},
%%            postcode={}, 
%%            state={},
%%            country={}}
%% \fntext[label3]{}

\title{Supplementary Material} %% Article title

%% use optional labels to link authors explicitly to addresses:
%% \author[label1,label2]{}
%% \affiliation[label1]{organization={},
%%             addressline={},
%%             city={},
%%             postcode={},
%%             state={},
%%             country={}}
%%
%% \affiliation[label2]{organization={},
%%             addressline={},
%%             city={},
%%             postcode={},
%%             state={},
%%             country={}}

% \author[label1]{Spencer Wadsworth} %% Author name
% \author[label1]{Jarad Niemi}

%% Author affiliation
% \affiliation[label1]{organization={Iowa State Univerity, Department of Statistics},%Department and Organization
%             addressline={2438 Osborn Dr}, 
%             city={Ames},
%             postcode={50011}, 
%             state={Iowa},
%             country={USA}}

%%Graphical abstract
% \begin{graphicalabstract}
%\includegraphics{grabs}
% \end{graphicalabstract}

%%Research highlights
% \begin{highlights}
% \item Research highlight 1
% \item Research highlight 2
% \end{highlights}
% 
%% Keywords

% 
% \end{frontmatter}

%% Add \usepackage{lineno} before \begin{document} and uncomment 
%% following line to enable line numbers
%% \linenumbers

%% main text
%%

\Huge
\begin{center}
  Supplementary Material
\end{center}
\normalsize





The figures herein are supplementary to those in the main manuscript. Figure 
\ref{fig:norm_dens} shows examples of marginal
posterior densities for QM of quantiles for a normal distribution from the 
simulation study in section 4.1 in
the main manuscript. The parameter uncertainty of the QGP and ORD models are 
unsurprisingly similar in
most cases, and where they differ it is hard to say that one is estimating the 
true parameter better than the
other. The IND model on the other hand has tighter densities, and often the 
bulk of the posterior is far from
the true parameter.


\begin{figure}
\centering
  \includegraphics[width = 1.1\linewidth]{Images/norm_params_fit.png}
\caption{Density plots of posterior distribution samples for normal parameters 
by QM for QGP, ORD, and
IND models. QM was done on estimated quantiles from a normal distribution with 
mean 4 and standard
deviation 3.5. The posterior densities are for $\mu$ (top left), 
$\sigma$ (top right), and 
sample size n (bottom). Plots are
faceted by the sample size $n \in \{50, 150, 500, 1,000\}$ 
($y$-axis) and number of 
quantiles $K \in \{7, 15, 23\}$ ($x$-axis).
Vertical lines (black) show the value of the true parameter}
\label{fig:norm_dens}
\end{figure}


Figures \ref{fig:exp_dens} and \ref{fig:exp_cov_dists} are from a simulation 
study similar to that of section 4.1 
except that QM was performed on
quantiles estimated from draws from an exponential distribution with 
parameter $\lambda = 4$ rather than a normal distribution. 
The results in the study for the exponential distribution were similar 
to those for the normal
distribution.


\begin{figure}
\centering
  \includegraphics[width = 1.1\linewidth]{Images/exp_params_fit.png}
\caption{Density plots of posterior distribution samples for the exponential 
parameters by QM for QGP,
ORD, and IND models. QM was done on estimated quantiles from a exponential 
distribution with parameter
$\lambda = 4$. The posterior densities are for $\lambda$
(left) and sample size $n$ (right). 
Plots are faceted by true sample
size $n \in \{50, 150, 500, 1,000\}$ ($y$-axis) and number of quantiles 
$K \in \{7, 15, 23\}$ 
($x$-axis). Vertical lines (black)
show the value of the true parameter.}
\label{fig:exp_dens}
\end{figure}






\begin{figure}[hbt!]
% \begin{subfigure}
  \includegraphics[width=\linewidth]{Images/exp_cov_dists.png}

\caption{ Posterior coverage (top) calculated as the percentage of times the 
true parameter fell within the
modeled 90\% credible interval over the 500 replications. Coverage is faceted 
by the exponential parameter $\lambda$
and $n$ with $K = 23$, and by increasing sample size ($x$-axis). 
The five models 
QGP, ORD, QGPN, ORDN,
and IND are colored as shown the legend. The horizontal line (black) is at 
the nominal 90\% level. Only QGP
and ORD appear for the parameter $n$ as they are the only two which estimate 
an unknown $n$. (bottom)
Distance between the true distribution and the estimated QM predictive 
distribution averaged over the 500
replications. Distances include UWD1, TV, and KLD for $K = 23$, 
and by increasing sample size (x-axis)}
\label{fig:exp_cov_dists}
\end{figure}



??Figures 11, 12, and 13 were referred to
in section 4.2 of the main manuscript.??





\begin{figure}[hbt!]
\centering
%\begin{subfigure}{.5\textwidth}
  \centering
  \makebox[\textwidth][c]{%
        \adjustbox{trim=0 0 0 0,clip}{\includegraphics[width=1.3\textwidth]{Images/quants_dens_la.png}}%
    }

\caption{QM fits of $K=23$ quantiles by KDE, SPL, IND, and QGP for $n \in \{50, 150, 500, 1{,}000\}$. The quantiles were sampled from the Laplace distribution $La(0,1)$. The quantile fits (left) show the true quantiles (black) with either the QM fit line (grey) or the credible intervals of 50\% (red) and 95\% (pink). 
The estimated PDF plots (right) show the true PDF (black) with either a the QM estimated PDF (grey) or the credible intervals of 50\% (red) and 95\% (pink).}
\label{fig:lp_fits}
\end{figure}


\begin{figure}[hbt!]
\centering
%\begin{subfigure}{.5\textwidth}
  \centering
  \makebox[\textwidth][c]{%
        \adjustbox{trim=0 0 0 0,clip}{\includegraphics[width=1.3\textwidth]{Images/quants_dens_gmix.png}}%
    }
\caption{QM fits of $K=23$ quantiles by KDE, SPL, IND, and QGP for $n \in \{50, 150, 500, 1{,}000\}$. The quantiles were sampled from the two component normal mixture distribution $w N(-1, 0.9) + (1-w)N(1.2, .6)$ where $w = 0.35$. The quantile fits (left) show the true quantiles (black) with either the QM fit line (grey) or the credible intervals of 50\% (red) and 95\% (pink). 
The estimated PDF plots (right) show the true PDF (black) with either a the QM estimated PDF (grey) or the credible intervals of 50\% (red) and 95\% (pink). }
\label{fig:gmix_fits}
\end{figure}


\begin{figure}[hbt!]
\centering
  \includegraphics[]{Images/mix_comps.png}
\caption{Future caption}
\label{fig:mix_comps}
\end{figure}


% \begin{table}[hbt!]
% \footnotesize
% \centering
% \caption{UWD1}
% \begin{adjustbox}{center}
% \begin{tabular}{lcccccc|cccccc}
%  & \multicolumn{6}{c}{QGP} & \multicolumn{6}{c}{QGP-n} \\
% \cmidrule(lr){2-7} \cmidrule(lr){8-13}
% Components & 1 & 2 & 3 & 4 & 5 & 6 & 1 & 2 & 3 & 4 & 5 & 6 \\
% \midrule
% EV & 0.063 & 0.025 & 0.021 & 0.020 & 0.020 & 0.020 & 
%       0.064 & 0.025 & 0.021 & 0.021 & 0.020 & 0.020\\
% La & 0.044 & 0.022 & 0.020 & 0.020 & 0.020 & 0.020 & 
%       0.044 & 0.021 & 0.020 & 0.020 & 0.020 & 0.020 \\
% MIX & 0.107 & 0.018 & 0.019 & 0.019 & 0.019 & 0.019 & 
%       0.157 & 0.018 & 0.019 & 0.018 & 0.018 & 0.018
% \end{tabular}
% \end{adjustbox}
% \end{table}












\begin{table}[hbt!]
\footnotesize
\centering
\caption{UWD1}
\begin{adjustbox}{center}
\begin{tabular}{clcccccc|cccccc}
& & \multicolumn{6}{c}{QGP} & \multicolumn{6}{c}{QGP-n} \\
\cmidrule(lr){3-8} \cmidrule(lr){9-14}
& Components & 1 & 2 & 3 & 4 & 5 & 6 & 1 & 2 & 3 & 4 & 5 & 6 \\
\midrule
\multirow{3}{*}{UWD1} &
EV & 0.063 & 0.025 & 0.021 & 0.020 & 0.020 & 0.020 & 
      0.064 & 0.025 & 0.021 & 0.021 & 0.020 & 0.020\\
 & La & 0.044 & 0.022 & 0.020 & 0.020 & 0.020 & 0.020 & 
      0.044 & 0.021 & 0.020 & 0.020 & 0.020 & 0.020 \\
 & MIX & 0.107 & 0.018 & 0.019 & 0.019 & 0.019 & 0.019 & 
      0.157 & 0.018 & 0.019 & 0.018 & 0.018 & 0.018 \\
      \hline
\multirow{3}{*}{TV} &
EV & 0.127 & 0.053 & 0.047 & 0.047 & 0.048 & 0.049 &
      0.125 & 0.052 & 0.039 & 0.039 & 0.039 & 0.039\\
 & La & 0.117 & 0.051 & 0.047 & 0.047 & 0.048 & 0.048 &
      0.117 & 0.048 & 0.043 & 0.043 & 0.043 & 0.042 \\
 & MIX & 0.226 & 0.038 & 0.033 & 0.032 & 0.033 & 0.034 &
      0.225 & 0.029 & 0.031 & 0.032 & 0.034 & 0.035 \\
      \hline
\multirow{3}{*}{KLD} &
EV & 0.171 & 0.025 & 0.014 & 0.013 & 0.014 & 0.015 & 
      0.170 & 0.025 & 0.011 & 0.010 & 0.010 & 0.010\\
& La & 0.110 & 0.014 & 0.012 & 0.012 & 0.012 & 0.012 & 
      0.111 & 0.013 & 0.010 & 0.010 & 0.010 & 0.010 \\
& MIX & 0.221 & 0.010 & 0.008 & 0.008 & 0.009 & 0.010 & 
      0.293 & 0.004 & 0.006 & 0.007 & 0.009 & 0.010
\end{tabular}
\end{adjustbox}
\end{table}






\end{document}


%% For citations use: 
%%       \citet{<label>} ==> Lamport (1994)
%%       \citep{<label>} ==> (Lamport, 1994)
%%
% Example citation, See \citet{lamport94}.

%% If you have bib database file and want bibtex to generate the
%% bibitems, please use
%%
\bibliographystyle{elsarticle-harv}
\bibliography{master_bib}

%% else use the following coding to input the bibitems directly in the
%% TeX file.

%% Refer following link for more details about bibliography and citations.
%% https://en.wikibooks.org/wiki/LaTeX/Bibliography_Management

% \begin{thebibliography}{00}
% 
% %% For authoryear reference style
% %% \bibitem[Author(year)]{label}
% %% Text of bibliographic item
% 
% \bibitem[Lamport(1994)]{lamport94}
%   Leslie Lamport,
%   \textit{\LaTeX: a document preparation system},
%   Addison Wesley, Massachusetts,
%   2nd edition,
%   1994.
% 
% \end{thebibliography}



\endinput
%%
%% End of file `elsarticle-template-harv.tex'.


