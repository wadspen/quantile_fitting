\documentclass{article}

\usepackage{enumerate}
\usepackage{hyperref}
\usepackage{amsmath}
% \usepackage{amsfonts}

\usepackage{xcolor}
\newcommand{\jarad}[1]{{\color{red} Jarad: #1}}
\newcommand{\spencer}[1]{{\color{blue} Spencer: #1}}

\begin{document}
  
  
\section*{Reviewer 1}


The paper presents a Gaussian process model for quantiles, called quantile GP, which is based on
quantile functions and sample quantiles. The QGP is estimated from a Bayesian perspective in
both simulation studies and a real data application. In particular, the authors employ the QGP to
approximate the distributions of quantile forecasts from the 2023-24 US Centers for Disease Control
collaborative flu forecasting initiative.


Below, you find my detailed comments.


\subsection*{Major Comments}

\begin{itemize}
  \item The introduction is weak and does not clearly communicate the contribution of the paper relative
  to the existing literature. It is difficult to understand the motivation for using a Quantile Gaussian
  process (QGP) over other methods such as an order statistics based model. I strongly recommend
  to completely rewrite the introduction to clearly explain the novelty of the paper and the key
  results.
  
  \item On page 10, when defining $\psi_{i,j}$ , should the subscript i and j enter in the right-hand side of the
  formula, particularly in $p$ and $p'$?
  
  \spencer{Fixed}

  \item In both the inferential and empirical parts, the authors refer to an unknown n. In what context
  is the sample size considered unknown?

  \item In Equation 16, what does $L$ represent? Moreover, what should $UWD1$ tells to the reader in
  comparison to $W_D$ or $p -W_D$?
  
  \spencer{Fixed}

  \item Regarding the different distance measures used, there seems to be redundancy. It may be better to
  avoid including all of them and instead consider incorporating other relevant metrics for evaluating
  quantiles and densities, such as the quantile CRPS (qCRPS, Gneiting, T., and Ranjan, R. (2011,
  JBES)) or the ACPS (Iacopini et al, (2023, JBES)).
  
  \spencer{Ignore}

  \item In Section 4.1.1, the choice of a truncated normal prior for $\sigma$ is unclear. Why did the authors not
  consider an Inverse Gamma or a Gamma prior? Moreover, how do the authors justify the choice
  of a prior center near the real data ($\mu = 4$ and prior $\mu \sim N(5, 7^2)$? For the prior on $n$, which
  appears to represent the sample size (assumed to be an integer), why is a Normal distribution
  used instead of a Uniform or Categorical distribution?
  
  \spencer{Responded}

  \item Although, the authors state that Bayesian inference is employed, it is unclear how $n$ estimated
  within the QGP framework. It would also be interested to present trace plots for the parameters
  and provide some convergence diagnostics.
  
  \spencer{Ignore}

  \item On page 17, what is $\theta^*_m$?
  
  \spencer{Clarified}
  
  \item When defining $\widehat{KLD}$, what does S denote?
  
  \spencer{Clarified}

  \item What is the main takeaway from Section 4.1.1? It appears that a simpler model (ORD) yields
  better results that the proposed QGP across different sample size and parameters estimators.
  
  \spencer{Responded}
  
  \item In Section 4.2, did the authors run some robustness check for different choices of C? Similarly to
  my previous point, why were the prior distribution not chosen as an inverse Gamma and centered
  differently? I assume that $\alpha$ = ($\alpha_1, . . . , \alpha_C$), is this correct? Furthermore, why are the results not
  compared with the best performing model from Section 4.1.1. i.e. the ORD model?
  
  \spencer{Responded}
  
  \item In the real data application, what is the value of $n$? Figure 5 and 6 are quite difficult to interpret.
  I was expecting a comparison with other methods, as done in the simulation study, potentially
  using metrics such as the qCRPS. Instead, only two metrics are compared, without explaining
  well what is happening to the dots and what are they referring to.
  \end{itemize}
  
  \spencer{Ignore}


\subsection*{Minor Comments}

  \begin{itemize}
  \item The writing is still conversational and the English is really poor and full of mistakes. For example,
  the authors use intensively the word “this” along the paper. Moreover, page 5 “the CDF and
  PDF are more used often for modeling and inference”; end of page 9, what does the sentence
  “But when these functions belong...” mean? Page 13, “to assess the how well...”; or “the the
  CDF ...”. On page 16 ,“decrase”
  
  \item Please include the formula along the text and not between dots. F.e. “the p-WD in terms of the
  corresponding QFs is defined in (15)”. should become “ the $p-WD$ in terms of the corresponding
  QFs is defined as...”. It is an example, but mainly all the equation needs to follow a similar
  specification.

  \end{itemize}
  
  
\section*{Reviewer 2}


I find the idea and motivation behind QGP compelling, and particularly 
attractive is that the general framework and the Stan code provided allow for 
flexible and relatively easy extension to other distributions beyond normal 
mixtures. The manuscript is coherent, clear, and well-structured. I believe 
there are some points that need further attention to enhance clarity.

\begin{itemize}

\item Clarification on the asymptotic nature of QGP:
 Since the QGP method relies on asymptotic results, I recommend explicitly 
 stating this in the abstract for clarity. Additionally, a concise summary in 
 the main text about when these asymptotic assumptions hold would be helpful. 
 While graphical results are provided, a practical guide on suitable sample 
 sizes or conditions would enhance the method's applicability.

\item Additional visual evaluation of estimated distributions:
 While thorough quantitative evaluations (KLD, Wasserstein) are provided, 
 additional graphical comparisons between the estimated and true underlying 
 distributions would offer valuable insight. After closely examining Figure 2, 
 it remains unclear to me why certain alternative methods underperform. For 
 instance, I suspect that the low coverage observed for the IND method is due 
 to its underestimation of uncertainty. Visualizing a set of estimated 
 distributions for selected scenarios (such as n=50 or n=150) would help reveal 
 where and why specific methods exhibit inaccuracies or weaknesses. These 
 visualizations would also support a more intuitive understanding of the 
 magnitude of approximation error; for example, it is not immediately obvious 
 whether a KLD of 0.05 is large or small.
 
\spencer{Image that was in supplementary material is added to main manuscript}

\item Justification and selection of informative priors:
 Currently, the priors chosen for the parameters of the QGP method are generic 
 and identical across all examples, including the flu hospitalization case 
 study. Given the Bayesian nature of QGP, this aspect is critically important. 
 The thoughtful use of informative priors could substantially distinguish the 
 QGP model from competing methods (ORD, KDE, SPL). Specifically, incorporating 
 expert knowledge into the prior distributions, especially for practical 
 applications such as flu modeling, could lead to improved inference, 
 increased stability with smaller sample sizes, reduced uncertainty, and 
 better interpretability. It would strengthen the manuscript if the authors 
 explicitly discuss this point in the conclusions. In particular, it would be 
 valuable to explain how to select these priors, perform prior predictive 
 checks (see, for example, (1)), suggest context-specific informative priors, 
 and clarify how these choices could vary across different applications.
Reference:
(1) Betancourt, M. (2017). Principled Bayesian Workflow. Available at:
\href{https://betanalpha.github.io/assets/case_studies/principled_bayesian_workflow.html}{}

\spencer{Ignore for now}

\item Sensitivity to the choice of distribution within QGP:
 While the paper fixes the QGP distributional assumption to a mixture of four 
 normal components, it would be informative to study how sensitive the results 
 are to this modeling choice. For example, it would be valuable to investigate 
 how varying the number of normal components in the mixture affects the 
 performance of QGP.
 
 \spencer{Added statement and small analysis to supplementary materials}

\item Explicit clarification on practical advantages of QGP:
 The manuscript carefully benchmarks QGP against alternative approaches, yet 
 the conclusions remain relatively modest regarding its practical advantages. 
 It would be helpful to include in the final discussion a concise summary of 
 the specific scenarios or data conditions where QGP outperforms the other 
 methods. Additionally, I believe that QGP stands out in two important aspects: 
 it naturally provides explicit uncertainty through the Bayesian posterior, and 
 it allows for the straightforward incorporation of expert knowledge via 
 informative priors.
 
\end{itemize}
\section*{Reviewer 3}

The paper introduces a “Quantile Gaussian Process” (QGP) model for estimating a 
full continuous distribution from a finite set of quantiles. The authors
apply this method in the context of probabilistic forecasting, particularly for
epidemic modeling, and compare the QGP approach with other quantile-matching
(QM) methods using both simulation studies and real CDC flu forecast data.

There are, however, several substantive issues with the manuscript, as outlined 
below. For these reasons, I do not believe it is currently suitable for
publication in Computational Statistics and Data Analysis. That said, I do not
wish to diminish the authors’ contributions to this important and active area
of research. With further development, this line of work could be expanded
into two complementary manuscripts: (i) a review-style paper summarizing 
existing quantile matching methods, and (ii) a computationally focused methods
paper with deeper theoretical or algorithmic contributions. And of course, that
will require substantial work beyond what is presented in this paper. The final
application section also has the potential to form the basis of a strong 
domainspecific paper, particularly in the biomedical or public health context. 
On a
personal note, I found the manuscript engaging, well-detailed and informative,
and I learned a great deal from reading it.

\begin{enumerate}[1]

  \item While the paper is well-written and the methodology is carefully 
  implemented, the contribution feels incremental rather than fundamentally
  novel. It reads more like an extended methodological review or a pedagogical 
  exposition rather than a research paper that breaks new ground.
  
  \spencer{Ignore}
  
  \item If the authors intend to emphasize their methodological contribution, 
  they
  should clearly articulate it in the introduction and present it more directly
  in Section 2. The focus should be placed primarily on describing the
  proposed method, rather than extensively reviewing existing literature.
  While such a review is valuable in a review article, in a research paper,
  much of this background could be moved to the supplementary material.
  
  \item For a stand-alone methods paper, the simulation study should be more
  comprehensive, incorporating additional tables and clearer, more informative 
  plots. The current figures make it difficult to assess the relative
  performance of the models.
  
  \spencer{Ignore}
  
\end{enumerate}  


  

\end{document}