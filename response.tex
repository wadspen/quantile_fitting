\documentclass{article}

\usepackage{enumerate}
\usepackage{hyperref}
\usepackage{amsmath}
% \usepackage{amsfonts}

\usepackage{xcolor}
\newcommand{\jarad}[1]{{\color{red} Jarad: #1}}
\newcommand{\spencer}[1]{{\color{blue} Spencer: #1}}

\begin{document}
  
  
\section*{Reviewer 1}


The paper presents a Gaussian process model for quantiles, called quantile GP, which is based on
quantile functions and sample quantiles. The QGP is estimated from a Bayesian perspective in
both simulation studies and a real data application. In particular, the authors employ the QGP to
approximate the distributions of quantile forecasts from the 2023-24 US Centers for Disease Control
collaborative flu forecasting initiative.


Below, you find my detailed comments.


\subsection*{Major Comments}

\begin{itemize}
  \item The introduction is weak and does not clearly communicate the contribution of the paper relative
  to the existing literature. It is difficult to understand the motivation for using a Quantile Gaussian
  process (QGP) over other methods such as an order statistics based model. I strongly recommend
  to completely rewrite the introduction to clearly explain the novelty of the paper and the key
  results.
  
  \item On page 10, when defining $\psi_{i,j}$ , should the subscript i and j enter in the right-hand side of the
  formula, particularly in $p$ and $p'$?
  
  \spencer{Fixed}

  \item In both the inferential and empirical parts, the authors refer to an unknown n. In what context
  is the sample size considered unknown?

  \item In Equation 16, what does $L$ represent? Moreover, what should $UWD1$ tells to the reader in
  comparison to $W_D$ or $p -W_D$?
  
  \spencer{Fixed}

  \item Regarding the different distance measures used, there seems to be redundancy. It may be better to
  avoid including all of them and instead consider incorporating other relevant metrics for evaluating
  quantiles and densities, such as the quantile CRPS (qCRPS, Gneiting, T., and Ranjan, R. (2011,
  JBES)) or the ACPS (Iacopini et al, (2023, JBES)).
  
  \spencer{Ignore}

  \item In Section 4.1.1, the choice of a truncated normal prior for $\sigma$ is unclear. Why did the authors not
  consider an Inverse Gamma or a Gamma prior? Moreover, how do the authors justify the choice
  of a prior center near the real data ($\mu = 4$ and prior $\mu \sim N(5, 7^2)$? For the prior on $n$, which
  appears to represent the sample size (assumed to be an integer), why is a Normal distribution
  used instead of a Uniform or Categorical distribution?
  
  \spencer{Responded}

  \item Although, the authors state that Bayesian inference is employed, it is unclear how $n$ estimated
  within the QGP framework. It would also be interested to present trace plots for the parameters
  and provide some convergence diagnostics.
  
  \spencer{Ignore}

  \item On page 17, what is $\theta^*_m$?
  
  \spencer{Clarified}
  
  \item When defining $\widehat{KLD}$, what does S denote?
  
  \spencer{Clarified}

  \item What is the main takeaway from Section 4.1.1? It appears that a simpler model (ORD) yields
  better results that the proposed QGP across different sample size and parameters estimators.
  
  \spencer{Responded}
  
  \item In Section 4.2, did the authors run some robustness check for different choices of C? Similarly to
  my previous point, why were the prior distribution not chosen as an inverse Gamma and centered
  differently? I assume that $\alpha$ = ($\alpha_1, . . . , \alpha_C$), is this correct? Furthermore, why are the results not
  compared with the best performing model from Section 4.1.1. i.e. the ORD model?
  
  \item In the real data application, what is the value of $n$? Figure 5 and 6 are quite difficult to interpret.
  I was expecting a comparison with other methods, as done in the simulation study, potentially
  using metrics such as the qCRPS. Instead, only two metrics are compared, without explaining
  well what is happening to the dots and what are they referring to.
  \end{itemize}


\subsection*{Minor Comments}

  \begin{itemize}
  \item The writing is still conversational and the English is really poor and full of mistakes. For example,
  the authors use intensively the word “this” along the paper. Moreover, page 5 “the CDF and
  PDF are more used often for modeling and inference”; end of page 9, what does the sentence
  “But when these functions belong...” mean? Page 13, “to assess the how well...”; or “the the
  CDF ...”. On page 16 ,“decrase”
  
  \item Please include the formula along the text and not between dots. F.e. “the p-WD in terms of the
  corresponding QFs is defined in (15)”. should become “ the $p-WD$ in terms of the corresponding
  QFs is defined as...”. It is an example, but mainly all the equation needs to follow a similar
  specification.

  \end{itemize}
  
  
  

  

\end{document}