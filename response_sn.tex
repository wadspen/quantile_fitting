%Version 3.1 December 2024
% See section 11 of the User Manual for version history
%
%%%%%%%%%%%%%%%%%%%%%%%%%%%%%%%%%%%%%%%%%%%%%%%%%%%%%%%%%%%%%%%%%%%%%%
%%                                                                 %%
%% Please do not use \input{...} to include other tex files.       %%
%% Submit your LaTeX manuscript as one .tex document.              %%
%%                                                                 %%
%% All additional figures and files should be attached             %%
%% separately and not embedded in the \TeX\ document itself.       %%
%%                                                                 %%
%%%%%%%%%%%%%%%%%%%%%%%%%%%%%%%%%%%%%%%%%%%%%%%%%%%%%%%%%%%%%%%%%%%%%

%%\documentclass[referee,sn-basic]{sn-jnl}% referee option is meant for double line spacing

%%=======================================================%%
%% to print line numbers in the margin use lineno option %%
%%=======================================================%%

%%\documentclass[lineno,pdflatex,sn-basic]{sn-jnl}% Basic Springer Nature Reference Style/Chemistry Reference Style

%%=========================================================================================%%
%% the documentclass is set to pdflatex as default. You can delete it if not appropriate.  %%
%%=========================================================================================%%

%%\documentclass[sn-basic]{sn-jnl}% Basic Springer Nature Reference Style/Chemistry Reference Style

%%Note: the following reference styles support Namedate and Numbered referencing. By default the style follows the most common style. To switch between the options you can add or remove ?Numbered? in the optional parenthesis. 
%%The option is available for: sn-basic.bst, sn-chicago.bst%  
 
%%\documentclass[pdflatex,sn-nature]{sn-jnl}% Style for submissions to Nature Portfolio journals
%%\documentclass[pdflatex,sn-basic]{sn-jnl}% Basic Springer Nature Reference Style/Chemistry Reference Style
\documentclass[pdflatex,sn-mathphys-num]{sn-jnl}% Math and Physical Sciences Numbered Reference Style
%%\documentclass[pdflatex,sn-mathphys-ay]{sn-jnl}% Math and Physical Sciences Author Year Reference Style
%%\documentclass[pdflatex,sn-aps]{sn-jnl}% American Physical Society (APS) Reference Style
%%\documentclass[pdflatex,sn-vancouver-num]{sn-jnl}% Vancouver Numbered Reference Style
%%\documentclass[pdflatex,sn-vancouver-ay]{sn-jnl}% Vancouver Author Year Reference Style
%%\documentclass[pdflatex,sn-apa]{sn-jnl}% APA Reference Style
%%\documentclass[pdflatex,sn-chicago]{sn-jnl}% Chicago-based Humanities Reference Style

%%%% Standard Packages
%%<additional latex packages if required can be included here>

\usepackage{graphicx}%
\usepackage{multirow}%
\usepackage{amsmath,amssymb,amsfonts}%
\usepackage{amsthm}%
\usepackage{mathrsfs}%
\usepackage[title]{appendix}%
\usepackage{xcolor}%
\usepackage{textcomp}%
\usepackage{manyfoot}%
\usepackage{booktabs}%
\usepackage{algorithm}%
\usepackage{algorithmicx}%
\usepackage{algpseudocode}%
\usepackage{listings}%
%%%%


%%%%%%Added by Spencer
\usepackage{dsfont}
\newcommand{\1}[1]{\mathds{1}\left[#1\right]}
\usepackage{adjustbox}
% \usepackage{biblatex}


%%%%%=============================================================================%%%%
%%%%  Remarks: This template is provided to aid authors with the preparation
%%%%  of original research articles intended for submission to journals published 
%%%%  by Springer Nature. The guidance has been prepared in partnership with 
%%%%  production teams to conform to Springer Nature technical requirements. 
%%%%  Editorial and presentation requirements differ among journal portfolios and 
%%%%  research disciplines. You may find sections in this template are irrelevant 
%%%%  to your work and are empowered to omit any such section if allowed by the 
%%%%  journal you intend to submit to. The submission guidelines and policies 
%%%%  of the journal take precedence. A detailed User Manual is available in the 
%%%%  template package for technical guidance.
%%%%%=============================================================================%%%%

%% as per the requirement new theorem styles can be included as shown below
\theoremstyle{thmstyleone}%
\newtheorem{theorem}{Theorem}%  meant for continuous numbers
%%\newtheorem{theorem}{Theorem}[section]% meant for sectionwise numbers
%% optional argument [theorem] produces theorem numbering sequence instead of independent numbers for Proposition
\newtheorem{proposition}[theorem]{Proposition}% 
%%\newtheorem{proposition}{Proposition}% to get separate numbers for theorem and proposition etc.

\theoremstyle{thmstyletwo}%
\newtheorem{example}{Example}%
\newtheorem{remark}{Remark}%

\theoremstyle{thmstylethree}%
\newtheorem{definition}{Definition}%

\raggedbottom
%%\unnumbered% uncomment this for unnumbered level heads

\begin{document}

\title[Article Title]{Response to Reviewers}

\Huge
\noindent Associate Editor
\\~\\
\normalsize

\emph{
The paper has been read by two referees and myself. The review team finds that 
the paper is novel and well-written. The authors should revise their work 
taking account of the comments of the referees. Of particular interest is 
exploring sensitivity to the number of quantiles and misspecification of the 
parametric model, and explaining details of the fitting in the case of finite 
mixture models.}
\\~\\

\Huge
\noindent Reviewer 1
\normalsize

\section*{Overview}
\emph{
I have carefully reviewed the manuscript. The authors propose to estimate the 
entire probability distribution of a random variable given a finite set of 
estimated quantiles. To do so, the authors leverage the quantile
central limit theorem in order to motivate their so-called Quantile Gaussian 
Process Model. Essentially, they
use a Gaussian process to model the finite set of quantiles, with the mean 
and covariance functions given by
a pre-specified distribution for the data. This enables quantile prediction 
and posterior predictive sampling
through the probability integral transform, which enables both CDF and PDF 
estimation. They explore
ways to incorporate this framework for nonparametric distribution functions. 
An extensive simulation study
compares both the estimation accuracy and the computational efficiency of 
their approach relative to other
Quantile Matching approaches. A real data analysis provides a new evaluation 
of quantile forecasts for flu
hospitalizations.
}

\emph{
I think this is a nice paper; it is well-written and provides some interesting 
ideas about estimating
distributions with coarse information. However, I think there is some room 
for additional detail and methods
development. I elaborate in further detail below,
}

\section*{Sensitivity Analyses are Needed}
\emph{
The effectiveness of the author’s method both for parameter and distribution 
estimations seems to rely on
the number of quantiles given at the onset. It seems that the authors conduct 
their simulations with $K = 23$
quantiles because it matches their real data analysis. However, for the 
method to be sufficiently general, it
behooves the authors to discuss the sensitivity of inferences and estimates 
when there are fewer quantiles
available. Obviously, the QGP won’t work well when there are very few 
quantiles available, but I encourage
the authors to look more into the quantiles available - utility frontier, 
perhaps with further simulations.
}


\section*{Finite Sample Behavior vs. Asymptotic Behavior?}
\emph{
I am confused at some of the coverage results presented in the simulation 
study. For example, looking at
Figure 5, the empirical coverage rates of the QGP when assuming a Gaussian 
mixture all approach 100%.
By contrast, when the marginal distribution is correctly specified, the 
intervals under-cover (and severely so
for small sample sizes) the true parameter values (e.g., Figure 2).
This result is counter-intuitive and should be explored more. One would 
think that coverage would close
the enumerated level when the model is correctly specified. When the model 
is misspecified, the intervals
should be biased and thus decrease coverage. The results presented in Figures 
2 and 5 tell the opposite story. Why is this?
}

\section*{Unknown Marginals}
\emph{
I appreciate that the authors use a Finite Gaussian Mixture when the marginal 
model is not known.
Further, their PIT transformed QGP renders model estimation relatively simple. 
This approach seems to be the primary use of the method, as often a more 
flexible distribution function will be needed in real data
analysis. However, I have a few questions/suggestions for the authors to 
consider:
}

\begin{itemize}
  
  \item \emph{Rather than pre-specifying the number of mixture components when
  they use
  the finite mixture as the
  marginal model, is it possible to learn the number of mixture components
  during model estimation,
  perhaps using a truncated Dirichlet Process Prior on the mixing weights?
  How will this impact computation? If the authors are able to do allow for
  an unknown number of mixture components, this
  will really strengthen the comparison to ORD. The results between the two
  methods are quite similar,
  though I note that the authors demonstrate the potential computational
  advantages of QGP relative
  to ORD.
  }

\item \emph{Can the authors explain how they do parameter estimation
(i.e. more means
and variances/covariances)
under the QGP when the finite mixture is assumed?
}

\item \emph{Why are there spikes in the estimated density functions in Figure 4
for some of the sample sizes, but
not all? I would assume that the spikes would smooth out as the sample sizes
get bigger, but this is
not the case.
}
\end{itemize}


\section*{Minor Points}

\begin{itemize}
 
  \item \emph{Feldman and Kowal 2024, JMLR consider the problem of estimating 
  distribution 
  functions from a
  finite set of quantiles and use smoothing techniques, and so should be cited 
  in the introduction
  }
  
  
  \item \emph{Can the authors clarify whether they performed QM using the QGP for all 
  160,000 forecasts?
  }
  
  \item \emph{Pg. 21, please reformat: “Unsurprisingly, the correlation between the QM 
  CRPS and WIS is very
  highly correlated for each method”
  }
\end{itemize}
\newpage

\Huge
\noindent Reviewer 2
\normalsize

\emph{
The paper proposes the Quantile Gaussian Process model for recovering a 
continuous distribution from a set of estimated quantiles. The approach 
builds on the well-known asymptotic distribution of sample quantiles, treating 
the vector of quantiles as approximately multivariate normal with a covariance 
structure determined by the quantile density function. The Bayesian 
formulation allows for uncertainty quantification and incorporation of prior 
information, which is useful in applications where one has domain knowledge 
about the underlying distribution.
}

\emph{
The theoretical development is clear and the connection to standard linear 
regression in the location-scale case is a nice insight that makes the model 
accessible. The alternative formulation using the probability integral 
transform is a practical contribution that extends the method to situations
where the quantile function is difficult to evaluate but the cumulative 
distribution function is available. This is demonstrated convincingly with the 
generalized lambda and metalog distributions, where the order statistics 
approach struggles computationally.
}

\emph{
The simulation studies are thorough. The comparisons with spline 
interpolation, kernel density estimation, the independent error model, and 
the order statistics model are fair and well executed. The results show that 
the QGP performs comparably to the order statistics model for standard 
distributions while offering advantages for quantile-defined distributions. 
The coverage analysis demonstrates that the asymptotic approximation works 
well even for moderate sample sizes.
}

\emph{
The application to the CDC FluSight forecasts is relevant and timely. 
Disease forecasting hubs increasingly rely on quantile submissions, and the 
ability to convert these to full distributions for scoring with the 
continuous ranked probability score or for ensemble construction is 
practically valuable. The analysis of over 160,000 forecasts shows that the 
method scales to real applications.
}

\emph{
One area where the paper could say more is the sensitivity of the QGP to model 
misspecification when using mixture distributions to approximate unknown 
shapes. The paper reports that four components work well for the 
distributions considered, but some discussion of how one might detect 
inadequacy in practice would be helpful for users applying this to new problems.
}

\emph{
The writing is clear throughout and the paper is well organized. The code 
availability will support adoption by practitioners working with forecast 
hubs and similar applications.
}


\end{document}
