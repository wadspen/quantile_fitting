\documentclass[pdflatex,sn-mathphys-num]{sn-jnl}


\usepackage{amsmath,amsthm,amssymb,scrextend,mathtools}
\usepackage{hyperref}
\usepackage{newtxmath} %for making indicator function
\usepackage{booktabs}
\usepackage{float} %hopefully keeps images and text in order
% \RequirePackage[colorlinks,citecolor=blue,urlcolor=blue,backref=page,backref=page]{hyperref}
%% The amsthm package provides extended theorem environments
%% \usepackage{amsthm}

%% The lineno packages adds line numbers. Start line numbering with
%% \begin{linenumbers}, end it with \end{linenumbers}. Or switch it on
%% for the whole article with \linenumbers.
%% \usepackage{lineno}

\newtheorem{theorem}{Theorem}
\newtheorem{lemma}{Lemma}
\newtheorem{definition}{Definition}
\usepackage{natbib}
\usepackage{adjustbox}
\usepackage{multirow}

% \journal{Computational Statistics and Data Analysis}

\begin{document}



\title{Supplementary Material} %% Article title



\Huge
\begin{center}
  Supplementary Material
\end{center}
\normalsize
 

\section*{Additional simulation study results to section 4.1}
Figure
\ref{fig:norm_dens} shows examples of marginal
posterior densities for QM of quantiles for a normal distribution from the
simulation study in section 4.1 in
the main manuscript. The parameter uncertainty of the QGP and ORD models are
unsurprisingly similar in
most cases, and where they differ it is hard to say that one is estimating the
true parameter better than the
other. The IND model on the other hand has tighter densities, and often the
bulk of the posterior is far from
the true parameter.


\begin{figure}[hbt!]
\centering
  \includegraphics[width = 1\linewidth]{Images/norm_params_fit.png}
\caption{Density plots of posterior distribution samples for normal parameters 
by QM for QGP, ORD, and
IND models. QM was done on estimated quantiles from a normal distribution with 
mean 4 and standard
deviation 3.5. The posterior densities are for $\mu$ (top left), 
$\sigma$ (top right), and 
sample size n (bottom). Plots are
faceted by the sample size $n \in \{50, 150, 500, 1,000\}$ 
($y$-axis) and number of 
quantiles $K \in \{7, 15, 23\}$ ($x$-axis).
Vertical lines (black) show the value of the true parameter}
\label{fig:norm_dens}
\end{figure}

\newpage
Figures \ref{fig:exp_dens} and \ref{fig:exp_cov_dists} are from a simulation 
study similar to that of section 4.1 
except that QM was performed on
quantiles estimated from draws from an exponential distribution with 
parameter $\lambda = 4$ rather than a normal distribution. 
The results in the study for the exponential distribution were similar 
to those for the normal
distribution.


\begin{figure}[hbt!]
\centering
  \includegraphics[width = 1.1\linewidth]{Images/exp_params_fit.png}
\caption{Density plots of posterior distribution samples for the exponential 
parameters by QM for QGP,
ORD, and IND models. QM was done on estimated quantiles from a exponential 
distribution with parameter
$\lambda = 4$. The posterior densities are for $\lambda$
(left) and sample size $n$ (right). 
Plots are faceted by true sample
size $n \in \{50, 150, 500, 1,000\}$ ($y$-axis) and number of quantiles 
$K \in \{7, 15, 23\}$ 
($x$-axis). Vertical lines (black)
show the value of the true parameter.}
\label{fig:exp_dens}
\end{figure}






\begin{figure}[hbt!]
% \begin{subfigure}
  \includegraphics[width=\linewidth]{Images/exp_cov_dists.png}

\caption{ Posterior coverage (top) calculated as the percentage of times the 
true parameter fell within the
modeled 90\% credible interval over the 500 replications. Coverage is faceted 
by the exponential parameter $\lambda$
and $n$ with $K = 23$, and by increasing sample size ($x$-axis). 
The five models 
QGP, ORD, QGPN, ORDN,
and IND are colored as shown the legend. The horizontal line (black) is at 
the nominal 90\% level. Only QGP
and ORD appear for the parameter $n$ as they are the only two which estimate 
an unknown $n$. (bottom)
Distance between the true distribution and the estimated QM predictive 
distribution averaged over the 500
replications. Distances include UWD1, TV, and KLD for $K = 23$, 
and by increasing sample size (x-axis)}
\label{fig:exp_cov_dists}
\end{figure}




\section*{Additional simulation study results to section 4.2}

Figures \ref{fig:lp_fits} and \ref{fig:gmix_fits} show examples of QM fits
to Laplace and normal mixture distributions as described in section 4.2.




\begin{figure}[hbt!]
\centering
%\begin{subfigure}{.5\textwidth}
  \centering
  \makebox[\textwidth][c]{%
        \adjustbox{trim=0 0 0 0,clip}{\includegraphics[width=1.3\textwidth]{Images/quants_dens_la.png}}%
    }

\caption{QM fits of $K=23$ quantiles by KDE, SPL, IND, and QGP for $n \in \{50, 150, 500, 1{,}000\}$. The quantiles were sampled from the Laplace distribution $La(0,1)$. The quantile fits (left) show the true quantiles (black) with either the QM fit line (grey) or the credible intervals of 50\% (red) and 95\% (pink). 
The estimated PDF plots (right) show the true PDF (black) with either a the QM estimated PDF (grey) or the credible intervals of 50\% (red) and 95\% (pink).}
\label{fig:lp_fits}
\end{figure}


\begin{figure}[hbt!]
\centering
%\begin{subfigure}{.5\textwidth}
  \centering
  \makebox[\textwidth][c]{%
        \adjustbox{trim=0 0 0 0,clip}{\includegraphics[width=1.3\textwidth]{Images/quants_dens_gmix.png}}%
    }
\caption{QM fits of $K=23$ quantiles by KDE, SPL, IND, and QGP for $n \in \{50, 150, 500, 1{,}000\}$. The quantiles were sampled from the two component normal mixture distribution $w N(-1, 0.9) + (1-w)N(1.2, .6)$ where $w = 0.35$. The quantile fits (left) show the true quantiles (black) with either the QM fit line (grey) or the credible intervals of 50\% (red) and 95\% (pink). 
The estimated PDF plots (right) show the true PDF (black) with either a the QM estimated PDF (grey) or the credible intervals of 50\% (red) and 95\% (pink). }
\label{fig:gmix_fits}
\end{figure}




\newpage

\section*{QGP normal mixture components analysis}

This section shows an analysis which was made to determine the number of 
normal components to use in the distribution assigned for use in the QGP.
Figure \ref{fig:mix_comps} shows the UWD1, TV, and KLD distances between 
a QGP fit distributions and simulated quantiles averaged over 500 simulated
replicates. Table \ref{tab:mix_comps} shows the average values. The results 
show that while the larger the number of components, the closer is the fit 
distribution in terms of the given distances. However, the imrpovement after 
three or four components is not much making four components attractive 
for relatively good fit and for model simplicity.

\begin{figure}[hbt!]
\centering
  \includegraphics[width=.9\textwidth]{Images/mix_comps.png}
\caption{UWD1, TV, and KLD distances averaged over 500 simulation replicates
for QGP fits of quantile simulated from extreme value, Laplace, and
mixture distributions. The QGP with normal 
mixture distributions of 1 to 6 components were fit for both the cases where 
$n$ is not known (left) and $n$ is known.}
\label{fig:mix_comps}
\end{figure}








\newpage

\begin{table}[hbt!]
\footnotesize
\centering
\caption{UWD1, TV, and KLD distances averaged over 500 simulation replicates
for QGP fits of quantile simulated from extreme value, Laplace, and
mixture distributions. The QGP with normal 
mixture distributions of 1 to 6 components were fit for both the cases where 
$n$ is not known (left) and $n$ is known.}
\begin{adjustbox}{center}
\begin{tabular}{clcccccc|cccccc}
& & \multicolumn{6}{c}{QGP} & \multicolumn{6}{c}{QGP-n} \\
\cmidrule(lr){3-8} \cmidrule(lr){9-14}
& Components & 1 & 2 & 3 & 4 & 5 & 6 & 1 & 2 & 3 & 4 & 5 & 6 \\
\midrule
\multirow{3}{*}{UWD1} &
EV & 0.063 & 0.025 & 0.021 & 0.020 & 0.020 & 0.020 & 
      0.064 & 0.025 & 0.021 & 0.021 & 0.020 & 0.020\\
 & La & 0.044 & 0.022 & 0.020 & 0.020 & 0.020 & 0.020 & 
      0.044 & 0.021 & 0.020 & 0.020 & 0.020 & 0.020 \\
 & MIX & 0.107 & 0.018 & 0.019 & 0.019 & 0.019 & 0.019 & 
      0.157 & 0.018 & 0.019 & 0.018 & 0.018 & 0.018 \\
      \hline
\multirow{3}{*}{TV} &
EV & 0.127 & 0.053 & 0.047 & 0.047 & 0.048 & 0.049 &
      0.125 & 0.052 & 0.039 & 0.039 & 0.039 & 0.039\\
 & La & 0.117 & 0.051 & 0.047 & 0.047 & 0.048 & 0.048 &
      0.117 & 0.048 & 0.043 & 0.043 & 0.043 & 0.042 \\
 & MIX & 0.226 & 0.038 & 0.033 & 0.032 & 0.033 & 0.034 &
      0.225 & 0.029 & 0.031 & 0.032 & 0.034 & 0.035 \\
      \hline
\multirow{3}{*}{KLD} &
EV & 0.171 & 0.025 & 0.014 & 0.013 & 0.014 & 0.015 & 
      0.170 & 0.025 & 0.011 & 0.010 & 0.010 & 0.010\\
& La & 0.110 & 0.014 & 0.012 & 0.012 & 0.012 & 0.012 & 
      0.111 & 0.013 & 0.010 & 0.010 & 0.010 & 0.010 \\
& MIX & 0.221 & 0.010 & 0.008 & 0.008 & 0.009 & 0.010 & 
      0.293 & 0.004 & 0.006 & 0.007 & 0.009 & 0.010
\end{tabular}
\end{adjustbox}
\label{tab:mix_comps}
\end{table}






\end{document}


%% For citations use: 
%%       \citet{<label>} ==> Lamport (1994)
%%       \citep{<label>} ==> (Lamport, 1994)
%%
% Example citation, See \citet{lamport94}.

%% If you have bib database file and want bibtex to generate the
%% bibitems, please use
%%
% \bibliographystyle{sn-mathphys}
\bibliography{master_bib}





\endinput


